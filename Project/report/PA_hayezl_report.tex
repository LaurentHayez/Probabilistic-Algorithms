\documentclass[a4paper, 11pt]{scrartcl}

\usepackage[utf8]{inputenc}
\usepackage[T1]{fontenc}
\usepackage{graphicx, wrapfig}
\usepackage{geometry}
\geometry{top=3cm, left = 3cm, right=3cm, bottom=3cm}
\usepackage{lmodern}
\usepackage{fancyhdr}
\usepackage{color, colortbl}
\usepackage{booktabs}
\usepackage[usenames, svgnames]{xcolor}
\usepackage{amsmath, amssymb, mathrsfs, amsthm, thmtools}
\usepackage[framemethod=tikz]{mdframed}
\usepackage{pgf, pgfplots, tikz}
\usepackage{hyperref}
\usepackage{makeidx}
\usepackage[inline]{enumitem}
\usepackage[english]{babel}


\newcommand{\N}{\mathbb{N}}
\newcommand{\Q}{\mathbb{Q}} 
\newcommand{\R}{\mathbb{R}}
\newcommand{\Z}{\mathbb{Z}} 
\newcommand{\C}{\mathbb{C}}
\newcommand{\K}{\mathbb{K}}
\renewcommand{\epsilon}{\varepsilon}
\renewcommand{\phi}{\varphi}
\renewcommand{\emph}{\textbf}



%%%%%%%%	Définitions des environnements des exercices	%%%%%%%%
%----- ENVIRONNEMENT POUR LES EXERCICES ----%
\declaretheoremstyle[
  spaceabove=0pt, spacebelow=0pt, headfont=\normalfont\bfseries\scshape,
    notefont=\mdseries, notebraces={(}{)}, headpunct={. }, headindent={},
    postheadspace={ }, postheadspace=4pt, bodyfont=\normalfont, %qed=$$,
    mdframed={
      leftmargin=-5,
      rightmargin=-5,
      hidealllines=true
   }
]{defstyle}

\declaretheorem[style=defstyle, title=Exercice]{ex}
%________________________________________________________


%----- ENVIRONNEMENT POUR LES SOLUTIONS ----%
\declaretheoremstyle[
  spaceabove=-10pt, spacebelow=0pt, headfont=\normalfont\scshape,
    notefont=\mdseries, notebraces={(}{)}, headpunct={. }, headindent={},
    postheadspace={ }, postheadspace=4pt, %bodyfont=\normalfont, 
    qed=\qedsymbol,
    mdframed={
      leftmargin=15,
      rightmargin=15,
      hidealllines=true,
      font=\small
   }
]{preuvestyle}

\declaretheorem[style=preuvestyle, numbered=no, title=Solution]{sol}

\synctex=1

\addtokomafont{disposition}{\normalfont\bfseries}

\title{\vspace{-1cm}\normalfont{\bfseries{Probabilistic Algorithms Project \\ {\Large Comparing heuristics for
      TSP}}}}
\author{Laurent \textsc{Hayez}}
\date{\today}% Date de création: 27 mars 2015\\ Dernière modification: \today}

\pagestyle{fancy}
\fancyhead[L]{Université de Neuchâtel}
\fancyhead[R]{Laurent Hayez}



\begin{document}


\renewcommand{\labelitemi}{\textbullet}

\maketitle

\thispagestyle{fancy}



\section{Introduction}

The Traveling Salesman Problem (TSP) is an old problem, having written sources as old as almost 300 years. The
problem consists of a person needing to visit $n$ cities, but with constraints. The constraints are that the
person must visit each of the $n$ cities only once, and must do it in an optimal way, i.e., the total
traveled distance must be the smallest possible.

This problem is hard to solve deterministically. Indeed, to find the optimal solution, we would need to test
every possible path in the graph composed of the $n$ cities, and the paths between them. If there are $n$
cities, the number of paths to test is $n!$, and the computations become impractical when $n$ is as small as
$20$. 

This hardness justifies the need to use heuristics, and probabilistic algorithms to solve the problem. Of
course, there is no guarantee that we will find the optimal path, but the heuristics return good enough
paths. The goal of this project is to compare a few heuristics that can be used to solve the TSP. We will
compare the best solutions found by each algorithms, the differences for the routes, the performance of the
algorithms, and pairwise statistical comparison of the algorithms. 

\section{Comparison of loss values}

Denote by $\mathcal{L}$ the sequence $\{L(\sigma_1^{\ast}), \ldots, L(\sigma_m^{\ast})\}$ of the best
solutions generated after calling each implemented algorithms $m = 30$ times. We will present tables showing
the minimum of $\mathcal{L}$, the maximum, the average and the $95\%$ condidence interval for this average for
each implemented algorithms.

\begin{table}[h]
\centering
\caption{Construction heuristics}
\label{table:construction-heuristics}
\begin{tabular}{@{}lllll@{}}
\toprule
Algorithms     & $\min(\mathcal{L})$ & $\max(\mathcal{L})$ & $\mathrm{mean}(\mathcal{L})$ & Confidence interval at 95\% \\ \midrule
best insertion & 1470.1821           & 1563.2285           & 1516.5628                    & {[}1506.6866, 1526.4391{]}  \\
shortest edge  & 1584.0766           & 1753.079            & 1671.8119                    & {[}1656.668, 1686.9558{]}   \\ \bottomrule
\end{tabular}
\end{table}


\begin{table}[h]
\centering
\caption{Improvement heuristics}
\label{table:improvement-heuristics}
\begin{tabular}{@{}lllll@{}}
\toprule
Algorithms  & $\min(\mathcal{L})$ & $\max(\mathcal{L})$ & $\mathrm{mean}(\mathcal{L})$ & Confidence interval at 95\% \\ \midrule
swap        & 1459.2188           & 1557.0646           & 1506.8777                    & [1497.346, 1516.4094]  \\
translation & 1443.1734           & 1513.6743           & 1483.3338                    & [1476.954, 1489.7137]   \\
inversion   & 1441.9932           & 1538.6809           & 1486.538                     & [1477.6003, 1495.4757]  \\
mixed       & 1417.9906           & 1497.824            & 1457.8618                    & [1451.398, 1464.3255]   \\ \bottomrule
\end{tabular}
\end{table}


\begin{table}[h]
\centering
\caption{Simulated annealing}
\label{table:simulated annealing}
\begin{tabular}{@{}lllll@{}}
\toprule
Algorithms  & $\min(\mathcal{L})$ & $\max(\mathcal{L})$ & $\mathrm{mean}(\mathcal{L})$ & Confidence interval at 95\% \\ \midrule
Metropolis  & 1484.3271           & 1570.1874           & 1520.7171                    & [1512.8956, 1528.5386]  \\
Heat Bath   & 1494.7218           & 1571.5438            & 1524.188                    & [1517.14, 1531.236]   \\ \bottomrule
\end{tabular}
\end{table}





	
\end{document}



%%% Local Variables:
%%% mode: latex
%%% TeX-master: t 
%%% End: